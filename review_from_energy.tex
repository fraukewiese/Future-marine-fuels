Reviewer #1: General comments

This paper is quite interesting, but many parametere have not been taken into consideration. It should be stressed more in the abstract that the prices represent an upward boundary, but also that it is not realistic that the IMO would opt for a carbon price only solution. Further, the approach to distribute emissions to Danish shipping vs other countries seems unimportant. Denmark cannot alone impose a carbon price that would decarbonize the sector. Further if Denmark were itself to impose a carbon price on emissions from the last port visited, that would be an incentive for ships to make an additional stop somewhere close to Denmark, undermining the carbon price. This would be the first form of carbon leakage. Is there no further danger of leakage to land based transport considered? 

"Liquified Bio Gas" (LBG) is not introduced in the text and should be further clarified. The biomass feedstock for biofuels should consider emissions from land use change and induced land use change, which may change the emissions calculation significantly in addition to methane slip. Or would this come from so called power to gas? 

1.	Introduction: 
Minor comment: "The ambition to reach climate pathways with limited overshoot of 1.5 ◦C requires global net zero CO2e emissions by 2050 [1]." 
-	The IPCC report talks about "In model pathways with NO OR limited overshoot of 1.5°C"

"So far, countries leave out international shipping in their energy and climate plans, leaving the responsibility to the International Maritime Organisation (IMO)."  
- Political precedent is really from the Kyoto Protocol more than anything. 
- Are "Nationally Determined Contribution" meant by "energy and climate plans"? Then use NDC. 

"their goal of 50% greenhouse gas reduction until 2050 of worldwide shipping [6] is neither ambitious enough to reach the goals of the Paris Agreement, nor underpinned with measures and possible pathway descriptions"
-	Reread IMO Initial GHG Strategy. This is a misrepresentation of the IMO goal: "to PEAK GHG emissions from international shipping AS SOON AS POSSIBLE and to reduce the total annual GHG emissions BY AT LEAST 50% by 2050 compared to 2008 whilst PURSUING EFFORTS TO PHASE THEM OUT as called for in the Vision as a point on a pathway of CO2 emissions reduction CONSISTENT with the Paris Agreement temperature goals."
-	Further target number 1 of the IMO Initial GHG strategy is " carbon intensity of the ship to decline through implementation of further phases of the energy efficiency design index (EEDI) for new ships" meaning that there will be command and control measures implemented in parallel to any market based mechanism. This will also affect necessary carbon prices. 

"Looking at a global warming potential over a 20-year time frame, a leakage rate of 3- 4 % would already use up the climate benefit of natural gas compared to coal, over a 100-year time frame it is 6-7 %"
-	Not that relevant as coal is not an alternative for shipping. Compare to HFO? 

LNG and methanol produced from natural gas is assessed to have the same order of magnitude as heavy fuel oil
-	LNG discussed in detail, but a bit more background on methanol would be helpful, it is not previously discussed here. 

Various options to reduce emissions are discussed in addition to fuel options. Though not mentioned in the article, some of these options come at negative costs ( would save the ship owner / ship operator money). Why are they note taken advantage of ? Depending on the barriers, it is unclear how much of a help a carbon price would be, even at higher levels.
 
According to the upper bound of predictions, future maritime transport emissions would increase by 250% in 2050 [40]. However, other predictions actually assume a decrease in transport demand due to less fossil fuel transported, as well as increased circular economy and effects from 3D-printing [41].
-	Would be helpful to have different scenarios with different levels of demand for maritime transport and to see how / if levels of demand affect the necessary carbon price. 
-	Different levels of demand will also be associated with different HFO fuel price levels. 

2. Materials and Methods
Costs include fuel, ship and infrastructure costs and are seen from a socio-economic perspective, excluding externalities. 
-       What are these externalities? GHG have a negative externality which is priced in the model. 

2.4.1 Fuel data (Page 7)
For fuels that are currently applied on a large scale - heavy fuel oil (HFO), marine diesel oil (MDO), bio-diesel oil (BDO) - the bunker index prices are used.
-	Including fuel futures for the shorter - 5 year term? 
-	The price of hydrogen (from electrolysis and renewable energy) is highly dependent on the price of renewable electricity in Denmark and Danish trade partners. If the price of renewable electricity rapidly falls, this will change the required carbon price to make hydrogen competitive. 

2.4.1 Emission budget

The apportioning of the global carbon budget to Denmark and the rest of the world is too simplistic. This does not seem to be a realistic or fair approach. Should examine more issues of equity. Further it is not realistic decarbonize only half a journey, either the whole route is decarbonized or none of it is. 

Page 9: the paper would greatly benefit from more detail in terms of the assumptions taken in each of the different scenarios. 

Page 13: For LBG, the development of the methane leakage problem is of outstanding importance
-	See comment above on LBG

Page 14: Compared to other sectors like electricity and heat, these mitigation costs per ton CO2e seem high, but one has to consider several points. First, our results can be interpreted as the upper bound of cost: Technologies considered are all applicable today, developments in other sectors applying similar technology options could further decrease costs and additional alternatives could evolve. Second, as shown in the demand reduction scenarios, any decrease in transport demand would save costs even beyond the proportional saving. This is due to not only fuel savings (proportional) but, also due to avoided investments in new ships (disproportionate). A decrease in transport demand especially has an effect in the period, in which new investments in ships are required and can thus avoid costs over-proportionally. And third, refits and hybrid solutions have not been considered to a great extent in the model functionality and could further decrease costs and ease the shift to different fuels. 
-	See comment above on hydrogen, the costs in the shipping sector (if assuming hydrogen or ammonia) are closely linked to technology progression of renewables in the electricity sector. 

What should next steps be from the point of view of Danish policy makers? For the IMO? The relevance of the text would benefit from clearer recommendations. 
