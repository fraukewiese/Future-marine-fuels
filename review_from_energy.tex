Dear Reviewer,
we would like to thank you very much for your valuable comments, which have significantly improved the paper. We have answered them one by one below.

Reviewer #1: General comments

This paper is quite interesting, but many parametere have not been taken into consideration. It should be stressed more in the abstract that the prices represent an upward boundary, but also that it is not realistic that the IMO would opt for a carbon price only solution. 
Answer: We have added that the price numbers indicate an upper boundary in the abstract. Regarding the likelihood of measures, we think that it is out of the scope of the paper to assess the interests of the institutions that could introduce measures. Our main focus is to provide an assessment for the option to climate neutral shipping from a technical and economical perspective and additionally mention regulative options. However, it is not in our scope to assess the potential role of the different stakeholders and the probability of measures to be applied.

Further, the approach to distribute emissions to Danish shipping vs other countries seems unimportant. Denmark cannot alone impose a carbon price that would decarbonize the sector. Further if Denmark were itself to impose a carbon price on emissions from the last port visited, that would be an incentive for ships to make an additional stop somewhere close to Denmark, undermining the carbon price. This would be the first form of carbon leakage. Is there no further danger of leakage to land based transport considered?
Answer: Thank you for that very valuable comment. We agree that it would not make sense for Denmark alone to impose a carbon price as it might have sounded in the abstract. However, although calculated for the Danish shipping, the numbers provide an indication of the upper boundary of a necessary worldwide carbon price. We have added a clarification to the discussion part.
Furthermore, we agree that a carbon price for Denmark alone would lead to the carbon leakage problems you mention. From our point of view it is still important to make an attempt to distribute international shipping emissions to countries. Otherwise they do not appear in National Climate Action Plans. This is currently the case, which results in these emissions being mostly out of the focus. But especially for countries with high contribution to maritime transport, scenarios and knowledge about potential future fuels are important since there can be essential synergies to land-based transport or other sectors. One example is the option to produce the required fuels in the country and use the excess heat from the process for district heating. In national scenarios, important inter-dependencies, potential synergies between the sectors (electricity, heat, fuels) or national barriers can be taken into account. For that, the national demand is of importance. Furthermore, since the efforts by IMO currently do not have the required ambition, we consider it important to get the discussion going about the responsibility for fair assignment of these emissions. Thus, scenarios from the Danish scope that include all emissions connected to the country are of value from our point of view.

"Liquified Bio Gas" (LBG) is not introduced in the text and should be further clarified. The biomass feedstock for biofuels should consider emissions from land use change and induced land use change, which may change the emissions calculation significantly in addition to methane slip. Or would this come from so called power to gas?
Answer: It would come from power to gas. We consider LBG as an potential future carbon-based electrofuel produced from carbon dioxide (CO2) and water using electricity as the primary energy source. We have added a description to the text.

1.	Introduction: 
Minor comment: "The ambition to reach climate pathways with limited overshoot of 1.5 ◦C requires global net zero CO2e emissions by 2050 [1]." 
-	The IPCC report talks about "In model pathways with NO OR limited overshoot of 1.5°C"
Answer: Thank you, this is quite and important and we have changed that in the text.

"So far, countries leave out international shipping in their energy and climate plans, leaving the responsibility to the International Maritime Organisation (IMO)."  
- Political precedent is really from the Kyoto Protocol more than anything.
Answer: We agree and still see reasons for countries to include a share of international transport emissions in their energy and climate plans. We have added these in the text: "Although there is no mechanism in the Kyoto protocol for countries to include their share of international transport in their Nationally Determined Contributions, we see two main reasons for taking these emissions also country-wise into account. First, due the high level of sector coupling between electricity, heat and fuels, national energy scenarios will differ if taking demand for future shipping fuels into account. Second, discussing potential pathways and fuel options for climate neutral shipping also in national energy plans contributes to progress in this area, supporting and calling for progress by the International Maritime Organisation (IMO)."
- Are "Nationally Determined Contribution" meant by "energy and climate plans"? Then use NDC.
Answer: We have added the NDC in the text, but still keep energy and climate plans, since we see climate and energy planning being made in the countries beyond the NDCs. We refer to these energy plans e.g. when talking about finding synergies in sector coupling etc.

"their goal of 50% greenhouse gas reduction until 2050 of worldwide shipping [6] is neither ambitious enough to reach the goals of the Paris Agreement, nor underpinned with measures and possible pathway descriptions"
-	Reread IMO Initial GHG Strategy. This is a misrepresentation of the IMO goal: "to PEAK GHG emissions from international shipping AS SOON AS POSSIBLE and to reduce the total annual GHG emissions BY AT LEAST 50% by 2050 compared to 2008 whilst PURSUING EFFORTS TO PHASE THEM OUT as called for in the Vision as a point on a pathway of CO2 emissions reduction CONSISTENT with the Paris Agreement temperature goals."
Answer: We understand your point and agree that the type of formulation might be too hard in the sense that a phaseout of GHG emissions until 2050 is not precluded by IMO's wording. However, we're convinced that a real commitment needs explicit targets, which go beyoned a 50% reduction in 2050 especially when it's based on a very transport intensive year, such as 2008.
-	Further target number 1 of the IMO Initial GHG strategy is " carbon intensity of the ship to decline through implementation of further phases of the energy efficiency design index (EEDI) for new ships" meaning that there will be command and control measures implemented in parallel to any market based mechanism. This will also affect necessary carbon prices.
Answer: Indeed, the EEDI would also affect the necessary carbon prices for a transition. As the result - GHG phase out in 2050 and compliance with the budget - is forced to be the same in the model, it's a question of the costs per measure, e.g fuel and engine switch or improved energy efficiency. In the model the parameter transport work capacity per energy is gradually increasing by 15% and can indirectly be interpreted as the result of implementing EEDI measures. However, this paper focused more on the fuel supply side to provide insight for energy system analysis, rather than on the demand side.

"Looking at a global warming potential over a 20-year time frame, a leakage rate of 3- 4 % would already use up the climate benefit of natural gas compared to coal, over a 100-year time frame it is 6-7 %"
-	Not that relevant as coal is not an alternative for shipping. Compare to HFO?
Answer: Thank you for pointing that out. We clarified that IEA gives numbers for electricity generation and added the number for comparison of LNG-HFO as ship fuel.

LNG and methanol produced from natural gas is assessed to have the same order of magnitude as heavy fuel oil
-	LNG discussed in detail, but a bit more background on methanol would be helpful, it is not previously discussed here.
Answer: We have added background information about methanol in the text.

Various options to reduce emissions are discussed in addition to fuel options. Though not mentioned in the article, some of these options come at negative costs ( would save the ship owner / ship operator money). Why are they note taken advantage of ? Depending on the barriers, it is unclear how much of a help a carbon price would be, even at higher levels.
Answer: This is a valid point. There is a large variety of reduction options discussed and the model could clearly benefit from their implementation. Many of them are though independent of the type of fuel used and would therefore not impinge the qualitative results of the fuel bench marking as such.
 
According to the upper bound of predictions, future maritime transport emissions would increase by 250% in 2050 [40]. However, other predictions actually assume a decrease in transport demand due to less fossil fuel transported, as well as increased circular economy and effects from 3D-printing [41].
-	Would be helpful to have different scenarios with different levels of demand for maritime transport and to see how / if levels of demand affect the necessary carbon price.
Answer: Till? Auf das demand reduced scenario verweisen!
-	Different levels of demand will also be associated with different HFO fuel price levels.
Answer: A marked clearing method is not implemented in the model. The costs are independent of the quantity demanded.

2. Materials and Methods
Costs include fuel, ship and infrastructure costs and are seen from a socio-economic perspective, excluding externalities. 
-       What are these externalities? GHG have a negative externality which is priced in the model.
Answer: We agree and have deleted "excluding externalities" in the text.

2.4.1 Fuel data (Page 7)
For fuels that are currently applied on a large scale - heavy fuel oil (HFO), marine diesel oil (MDO), bio-diesel oil (BDO) - the bunker index prices are used.
-	Including fuel futures for the shorter - 5 year term?
Answer: Fuel futures are not included. From what we learned when talking to an expert in liner shipping (Torben Anker Sørensen) shipping companies usually secure their prices for 6 months ahead.
-	The price of hydrogen (from electrolysis and renewable energy) is highly dependent on the price of renewable electricity in Denmark and Danish trade partners. If the price of renewable electricity rapidly falls, this will change the required carbon price to make hydrogen competitive.
Answer: We agree. This effect is represented in the threshold analysis, where all considered cost components for producing the different fuels are varied. Moreover, our analysis also implies that actually the new build ship costs could present the most influencing variable in the future.

2.4.1 Emission budget

The apportioning of the global carbon budget to Denmark and the rest of the world is too simplistic. This does not seem to be a realistic or fair approach.Should examine more issues of equity. 
Answer: We are aware of the fact that the "constant emission ratio" apportioning method might not be the most fair and we acknowledge that there are more elaborated ones as e.g. stated e.g. in DuPont2016 [https://iopscience.iop.org/article/10.1088/1748-9326/11/5/054005/meta]. Though, we favour this rather simple approach in order to increase the transparency behind the assumption. This paper doesn't intent to embark on the discussion of fair co2 budget allocation with regard to indicators as GDP, population, area or historical responsibility. In general, we see the Danish budget as an upper bound, as the Danish budget would probably be tightened when applying one of the other above mentioned methods.

Further it is not realistic decarbonize only half a journey, either the whole route is decarbonized or none of it is.
Answer: We agree. This method is not thought to be a representation of emission reduction in reality, but an accounting method. If all countries would apply this approach, all international emissions would be covered, this is from our point of view an advantage of the method.

Page 9: the paper would greatly benefit from more detail in terms of the assumptions taken in each of the different scenarios.
Answer: We have added references to tables in the appendix and supplementary material in the

Page 13: For LBG, the development of the methane leakage problem is of outstanding importance
-	See comment above on LBG
Answer: Till

Page 14: Compared to other sectors like electricity and heat, these mitigation costs per ton CO2e seem high, but one has to consider several points. First, our results can be interpreted as the upper bound of cost: Technologies considered are all applicable today, developments in other sectors applying similar technology options could further decrease costs and additional alternatives could evolve. Second, as shown in the demand reduction scenarios, any decrease in transport demand would save costs even beyond the proportional saving. This is due to not only fuel savings (proportional) but, also due to avoided investments in new ships (disproportionate). A decrease in transport demand especially has an effect in the period, in which new investments in ships are required and can thus avoid costs over-proportionally. And third, refits and hybrid solutions have not been considered to a great extent in the model functionality and could further decrease costs and ease the shift to different fuels. 
-	See comment above on hydrogen, the costs in the shipping sector (if assuming hydrogen or ammonia) are closely linked to technology progression of renewables in the electricity sector.
Answer: Schlaegt er vor, seinen Punkt zusaetzlich aufzufuehren oder wie verstehst Du das? So wuerde ich das verstehen. Kønnen wir gerne zusætzlich auffuehren, vielleicht noch vor dem dritten Punkt.

What should next steps be from the point of view of Danish policy makers? For the IMO? The relevance of the text would benefit from clearer recommendations. 
Answer: We definitely agree on that. An additional paragraph at the end of the paper discusses further recommendations.



