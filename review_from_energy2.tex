Reviewer #1: 
General comments 

- Represents improvements from the first draft. has benefited from further clarification.
Answer: Thank you for the positive feedback.

- could benefit from proofreading from a native English speaker. Minor grammatical issues.
Answer: We agree

- does the recent decision in the lastest MEPC to strengthen the EEDI affect the likley scenarios and assumptions?
Answer:

- could still benefit from a deeper dive and possible update of assumptions / progression of cost of renewable fuels based on renewable electricity as well as batteries. 
Answer: We agree that additional sensitivity analysis of the progression of renewable fuel costs would be very interesting. We however think that it is partly included in the analysis of varying the costs of each of the fuel option within a very wide range (0 to 100%). Thus, also for the cost for fuels based on renewable electricity we used a very wide cost range. We find that a deeper look just in part of the fuel option would cause an imbalance in detail for the paper. We thus would prefer to keep that deeper dive for an extra study.

Abstract
"Our modelling results indicate, that either strong regulative carbon budgets or a carbon price of :: at :::: most:350-450 e2016/t CO2e would be necessary to induce the urgent transition."

-	Carbon budgets or a carbon price are suggested as two possible alternatives. There are however major pros and cons to each. It is unclear if carbon budgets would lead to prices that would foster decarbonization - what of there is a trade war (US China for example) that temporarily reduces demand and supresses prices? The respective incentive to invest in new technology would be lost.  If the model comes up with prices, perhaps it would be best to come up with prices.
Answer: In our study, we do not want to include the discussion of the appropriateness of carbon budgets or carbon prices since it is a quite extensive discussion. 

-	A preference of an international CO2 quota scheme is also inserted at the end of the document with no explanation for why a quota scheme would be better than a tax or levy and no discussion of the major disadvantages of a quota scheme. DOUBTFUL ARGUMENT
Answer: We fully agree, we actually did not want to show a tendency towards quota or tax and have thus adapted it in the conclusion, mentioning different options.


"This would double today's average cargo transport costs, but increase average import values only by 6-8 %."

-	What about the effect on the average consumer's consumption basket? Would a consumer notice?
Answer: Depends highly on the transported good. Small impact for goods with high value and small volume and vice versa.

"Regarding fuel technologies, hydrogen, methanol and ammonia are most compatible from a socio-economic cost perspective."
-	Do you see no role for electrification? Even for shorter distances?
Answer: The engine would be electric in all cases, but the storage medium on board would just be batteries for short distances. We did not made it clear enough in the paper, we have added an explanation.

"Liquefied natural gas as an alternative intermediate solution would only have a short window of opportunity, due to methane leakage causing high greenhouse gas emissions as well as high fuel and technology costs."

-	Consider risk of fossil fuel lock in and stranded assets?
Answer: Important points, the problem of stranded assets is included in the modelling setup. The fossil fuel lock in is an additional argument against LNG which we did not include in the modelling (would be an integer problem) but it further stresses our point.

Introduction

 "Although transport is more challenging [3, 4], in recent years an increasing amount of studies"
-	Minor grammatical comment. "Number" of studies not "amount". Akin to fewer studies because studies are countable - see distinction between fewer and less.
Answer: Is changed.

"So far, countries leave out international shipping in their energy and climate plans"
-	Seems a bit confusing to say energy and climate plans as this is a specific regulatory measure in the EU for each individual member state. Consider climate and transport policy planning?
Answer: We agree that this is more precise and changed it in the text to your suggestion.

"Although::::: there::is::: no::::::::::: mechanism :: in:::: the:::::: Kyoto ::::::: protocol::: for::::::::: countries:: to::::::: include::::: their:::::: share :: of:::::::::::: international::::::::: transport :: in::::: their ::::::::: Nationally::::::::::: Determined::::::::::::: Contributions,::: we::: see:::: two::::: main::::::: reasons::: for:::::: taking::::: these :::::::: emissions:::: also:::::::::::: country-wise::::: into :::::::: account."

-	Nationally Determined Contributions are only under the Paris Agreement. Under the Kyoto Protocol there was no such thing.
Answer: Thank you for that important correction. Is changed in the text to Paris Agreement.

"its possibility to also reduce climate impact of shipping has increasingly been questioned"

-	Ability? Instead of possibility?
Answer: Yes, ability is more accurate and changed in the text.

"On the other hand, the implications for greenhouse gases (GHG) depend on how the natural gas is extracted, processed, distributed, and used"

-	Specifically mention "well to propeller assessment"?
Answer: Good idea, that is what we refer to. Changed in the text.

On Methanol - "::::::::: municipal ::::: waste::: or :::::::: biomass"
-	Would there be sufficient municipal waste to produce enough methanol? What would be the induced land use implications in terms of emissions for biomass on this scale?
Answer: The amount of municipal waste in the future and its availability for fuel production is quite unclear, but it would not be possible to base all methanol generation on waste. The same holds true for sustainable biomass. Thus, if methanol is THE option, methanol generation based on renewable electricity will be definitely required.


"A meta-study looking at measures and fuels of various studies [31] suggests that a 75% emission reduction is possible until 2050,"
-	Check use of "until" vs. "by".
Answer: Good point, we changed it in the text accordingly.

"However, other predictions actually assume a decrease in transport demand due to less fossil fuel transported,"
-	Cite these other predictions?
Answer:

3. Results and discussion 

Page 15 lines 19-22
 " Note, :::: that:::::::::: technology::::::::::: progression :: of
::::::::: renewable :::::: energy::::::: sources:: in::: the:::::::::: electricity ::::: sector::::: have:: a :::::: strong :::::::: inuence::: on ::: the
:::: costs::: in :::::::::: alternative::::: green::::::::: shipping :: (if::::::::: assuming::::::::: hydrogen::: or :::::::::: ammonia)."

-	Also battery costs. Is an S curve trend of battery market uptake and cost progression assumed?
Answer: It might be true that battery costs are having a large influence, but we cannot show that directly from our results. No market uptake included but an asymptotically declining price curve for battery propelled vessels.

Conclusion
-	See above comment in the abstract section on quotas vs. levy. 
Answer: See answer above.
